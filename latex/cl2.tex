\documentclass[11pt]{article}

\usepackage{sectsty}
\usepackage{graphicx}

% Margins
\topmargin=-0.45in
\evensidemargin=0in
\oddsidemargin=0in
\textwidth=6.5in
\textheight=10.0in
\headsep=0.25in

\title{Cover Letter}
\date{}

\begin{document}

I am writing to apply for the role of software developer at Science and Technology Facilities Council. I am currently based in Cyprus but I am open to relocating for the role. I have a total of 5 years of developer experience, primarily in the finance sector. As a child, I was always curious about the world. I loved to read and I liked to take things apart to learn how they worked (I did not always put them back together however). This led me to love science and mathematics, and to choose to study physics at the University of Warwick. At university, I took several introduction to programming modules (primarily in C), and a significant part of my final year project involved writing Python code.

Though I did not initially study computer science at university, I have since developed a passion for the subject. Over the years, I have spent time working through textbooks and watching lectures to learn some of the topics covered by a computer science degree, and to satisfy my curiosity generally. Additionally, I have always enjoyed solving puzzles, and, as such, I often spend time solving mathematics puzzles using programming. This gives me an opportunity to experiment with new languages and concepts.

For example, recently, I have followed a lecture series by Abelson and Sussman that covers Lisp programming and the Lisp language design, and I have worked through the textbook nand2tetris that takes you from building simple hardware to building a simple JVM-like language. I have also recently enjoyed experimenting with writing my own toy Lisp implementation to better understand how programming languages work. 

My exposure to multi-threaded/concurrent programming is from my own personal study, and I am confident in the concepts involved. Particularly, I enjoy using the concurrency model in the Go programming language which uses channels and goroutines. I have recently experimented with the pthreads library for C, and I have also worked through the textbook 'The Little Book of Semophores' that explains how semophores are the building block used to solve the problems of concurrent programming. 

As well as lower-level computer science concepts, I have spent time working on web interfaces using the Apache Tapestry web framework and writing Java code to generate pdf reports. I pride myself on my ability to create thoughtfully laid out documents and user interfaces.

My current role has been very satisfying for many reasons. Though I had experience from before, I am now very comfortable using git, bash, and other Linux command line tools. In my company, there is a strong focus on object-orientated programming. The book 'Clean Code' by Robert Martin is given to every new joiner and we follow the guidance there closely. Additionally, test-driven development is prioritized, which very much suits my style of programming. 

Despite the many positives in my current role, I have begun to look around for other opportunities to learn and grow. Although I have worked primarily with Java, for the past three years, I have also spent significant time debugging and maintaining our C code base. I find the challange of dealing with low-level languages particularly satisfying. Though it can sometimes be more time-consuming and technical, it can also give you a peace of mind that nothing is happening magically 'behind the scenes'; what you see is what you get. 

I find the prospect using my knowledge and experience in order to contribute to projects that further scientific progress incredibly exciting.

Thank you very much for considering my application and I look forward to hearing from you.

Best regards, Max.
%--/Paper--

\end{document}
